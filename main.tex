% Poss�veis TODO
% - expandir nas equacoes xx yy
\documentclass[a4paper,titlepage]{article}
\usepackage[latin1]{inputenc}
\usepackage[T1]{fontenc}
\usepackage[portuges]{babel}
% setspace - \doublespace \onehalfspace
% fullpage - ??
\usepackage{verbatim,url}
\usepackage{graphicx}
\usepackage[bf]{subfigure}
\usepackage[bf]{caption}
\usepackage{amsmath,amssymb}
\usepackage{algorithm,algorithmic}
\usepackage{mathtools,empheq}

\begin{document}

\begin{titlepage}
\renewcommand{\title}{%
  {\LARGE Revisão Comentada de Artigo}\\
  \mbox{Barycentric Subspace Analysis on Manifolds}%
}
\renewcommand{\author}{Juliana Santos Barcellos Chagas Ventura}
\renewcommand{\date}{\today}
\newcommand{\info}{%
  \raisebox{4pt}[-4pt]{%
  \includegraphics[height=1.3cm]{figs/logo-iprj.png} 
  \hspace{0.1in}
  }\\

  Instituto Politécnico -- IPRJ\\
  Universidade do Estado do Rio de Janeiro\\[2em]
  
  \includegraphics[height=1.3cm]{figs/logo-ppgmc.png}\\
  Programa de Pós Graduação em Modelagem Computacional\\
  Disciplina de Variedades Diferenciáveis\\
  prof. Ricardo Fabbri\\[1em]

  Nova Friburgo, \date\\[1.5cm]
}

%% Abstand zwischen oberem Blattrand und Titel.
\newlength{\topToTitle} 
\setlength{\topToTitle}{0pt}

%% Abstand zwischen linkem Blattrand und Titel.
\newlength{\leftToTitle} 
\setlength{\leftToTitle}{-60pt}

%% Abstand zwischen Titel und Info-Feld.
\newlength{\titleToInfo} 
\setlength{\titleToInfo}{10cm}

%% \myTextWidth erhoehen, um Info-Feld weiter nach Rechts zu schieben.
\newlength{\myTextWidth}
\setlength{\myTextWidth}{\textwidth}
\advance\myTextWidth by 1cm


\thispagestyle{empty}
\vspace*{\topToTitle}
\begin{minipage}{\myTextWidth}
  \sffamily
  \hspace*{\leftToTitle}\begin{minipage}{11cm}
    \Large\textbf{Trabalho final}\\[1.5cm]
    \title\\[1.5cm]
    \author
  \end{minipage}\\

  %% \enlargethispage{} um ggfs. Titel und Info-Feld weiter
  %% auseinanderziehen zu koennen.
  \vspace*{\titleToInfo}

  \begin{minipage}{\textwidth}
    \flushright
    \info
  \end{minipage}
\end{minipage}%
\end{titlepage}


\section{Preliminares}

Este trabalho consiste em expandir o artigo~\cite{Pennec:AnnStat:2018},
cuja primeira p�gina est� reproduzida na Figura~\ref{fig:paper:page1}.
O presente texto � uma vers�o comentada do artigo,
expandindo o m�ximo poss�vel os conceitos ligados a Variedades Diferenci�veis.
Desta forma, o presente texto � um super-conjunto do referido artigo.
Ele cont�m todo o artigo, possivelmente em forma de recortes, com expans�es e
coment�rios, e conex�es com outros conceitos.

\begin{figure}
\centering
\frame{\includegraphics[width=0.8\linewidth]{figs/pennec2018-page1.png}}
\caption{% 
Primeira p�gina do artigo sendo revisado. Para baixar, foi necessario o Scihub
pois o portal da CAPES nao continha este periodico apos 2017.
}\label{fig:paper:page1}
\end{figure}

Ser� utilizada uma mistura de l�nguas nesta revis�o, sendo o ingl�s preferido
sempre que poss�vel. Sendo assim, n�o teremos o trabalho de traduzir do ingl�s
algumas constru��es b�sicas do \LaTeX\ como \emph{Theorem} ou
\emph{Definition}.


\section{Tabela de Nota��o}

This section has a summary of the notation used in the paper.

\section{Commented and Expanded Abstract}

\section{Commented and Expanded Section 1: Introduction}

\section{Commented and Expanded Section 2: Riemannian geometry}

\section{Commented and Expanded Section 3: Exponential Barycentric Subspaces
(EBS) and affine spans}

\section{Commented and Expanded Section 4: Fr�chet/Karcher barycentric
subspaces}

\section{Commented and Expanded Section 5: Properties of the barycentric
subspaces}

\section{Commented and Expanded Section 6: Barycentric subspace analysis}

\section{Commented and Expanded Section 7: Discussion}

\section{Commented and Expanded Appendix: Proof of Theorem 8}

\section{Commented and Expanded Supplementary Material}




\section{Refer�ncias}
\bibliographystyle{ieeetr}
\bibliography{refs}
%bib/edge-linking,bib/deformable,bib/medical,bib/graphics,bib/texture,bib/imaging,bib/tracking,bib/shape-papers,bib/bib-header,bib/video,bib/math-books,bib/math,bib/psych-books,bib/metric,bib/edge,bib/leymarie_pami_scaffold,bib/vision-books,bib/vision,bib/nn-search,bib/multidimscaling,bib/psychophysics,bib/indexing,bib/segmentation,bib/image-databases,bib/shape-matching,bib/neuro,bib/skeleton,bib/skeleton2D,bib/aspect-graphs,bib/recognition,bib/surface-networks,bib/ridge,bib/proceedings,bib/perceptual-grouping,bib/continuation,bib/graph-matching-2,bib/cooper}
%\input{paper.bbl}

% Assinaturas:
%\newpage
%\ \\\vspace{7cm}
%\center $\overline{\ \ \ Ricardo\ Fabbri\ \ \ }$
%\ \\\vspace{4cm}
%\center $\overline{\ \ \ Luciano\ da\ Fontoura\ Costa\ \ \ }$
\end{document}

% Poss�veis TODO
% - expandir nas equacoes xx yy
\documentclass[a4paper,titlepage]{article}
\usepackage[latin1]{inputenc}
\usepackage[T1]{fontenc}
\usepackage[portuges]{babel}
% setspace - \doublespace \onehalfspace
% fullpage - ??
\usepackage{verbatim,url}
\usepackage{graphicx}
\usepackage[bf]{subfigure}
\usepackage[bf]{caption}
\usepackage{amsmath,amssymb}
\usepackage{algorithm,algorithmic}
\usepackage{mathtools,empheq}

\begin{document}

\begin{titlepage}
\begin{changemargin}{-0.5cm}{-1cm}
\renewcommand{\title}{%
  {\LARGE Revis�o Comentada de Artigo}\\
  \mbox{Barycentric Subspace Analysis}%
}
\renewcommand{\author}{Aluno Ono Nono}
\renewcommand{\date}{\today}
\newcommand{\info}{%
  \raisebox{4pt}[-4pt]{%
  \includegraphics[height=1.3cm]{figs/logo-iprj2.eps} 
  \hspace{0.1in}
  }\\

  Disciplina Variedades Diferenci�veis\\
  Instituto Polit�cnico -- IPRJ\\
  Universidade do Estado do Rio de Janeiro\\[1.5cm]

  Nova Friburgo, \date\\[1.5cm]

  \textbf{Projetos Relacionados}\\[4pt]
   FAPERJ Jovem Cientista do Nosso Estado E25/2014 204167\\
   FAPERJ APQ1 01/09335-8 -- coordenador\\
   FAPERJ E-26/010.002701/2014 -- coordenador\\
   FAPERJ E-26/202.647/2016 -- participa��o\\
   NSF Robust Intelligence Awards 1116140 \& 1319914\\
   UERJ Proci�ncia 2014--2017
}

%% Abstand zwischen oberem Blattrand und Titel.
\newlength{\topToTitle} 
\setlength{\topToTitle}{30pt}

%% Abstand zwischen linkem Blattrand und Titel.
\newlength{\leftToTitle} 
\setlength{\leftToTitle}{-50pt}

%% Abstand zwischen Titel und Info-Feld.
\newlength{\titleToInfo} 
\setlength{\titleToInfo}{7cm}

%% \myTextWidth erhoehen, um Info-Feld weiter nach Rechts zu schieben.
\newlength{\myTextWidth}
\setlength{\myTextWidth}{\textwidth}
\advance\myTextWidth by 1.5cm


\thispagestyle{empty}
\vspace*{\topToTitle}
\begin{minipage}{\myTextWidth}
  \sffamily
  \hspace*{\leftToTitle}\begin{minipage}{11cm}
    \Large\textbf{Trabalho final}\\[1.5cm]
    \title\\[1.5cm]
    \author
  \end{minipage}\\

  %% \enlargethispage{} um ggfs. Titel und Info-Feld weiter
  %% auseinanderziehen zu koennen.
  \vspace*{\titleToInfo}

  \begin{minipage}{\textwidth}
    \flushright
    \info
  \end{minipage}
\end{minipage}%
\end{changemargin}
\end{titlepage}


\section{Preliminares}

Este trabalho consiste em expandir o artigo~\cite{Pennec:AnnStat:2018},
Figura~\ref{fig:paper:page1} sendo
um super-conjunto do mesmo. Trata-se de uma vers�o comentada do manuscrito,
expandindo o m�ximo poss�vel os conceitos ligados a Variedades Diferenci�veis.

Ser� utilizada uma mistura de l�nguas nesta revis�o, sendo o ingl�s preferido
sempre que poss�vel. Sendo assim, n�o teremos o trabalho de traduzir do ingl�s
algumas constru��es b�sicas do \latex\ como \emph{Theorem} ou
\emph{Definition}.


\section{Nota��o}

This section has a summary of the notation used in the paper.

\section{Commented and Expanded Introduction of~\cite{Pennec:AnnStat:2018}}

\section{Refer�ncias}
\bibliographystyle{ieeetr}
\bibliography{refs}
%bib/edge-linking,bib/deformable,bib/medical,bib/graphics,bib/texture,bib/imaging,bib/tracking,bib/shape-papers,bib/bib-header,bib/video,bib/math-books,bib/math,bib/psych-books,bib/metric,bib/edge,bib/leymarie_pami_scaffold,bib/vision-books,bib/vision,bib/nn-search,bib/multidimscaling,bib/psychophysics,bib/indexing,bib/segmentation,bib/image-databases,bib/shape-matching,bib/neuro,bib/skeleton,bib/skeleton2D,bib/aspect-graphs,bib/recognition,bib/surface-networks,bib/ridge,bib/proceedings,bib/perceptual-grouping,bib/continuation,bib/graph-matching-2,bib/cooper}
%\input{paper.bbl}

% Assinaturas:
%\newpage
%\ \\\vspace{7cm}
%\center $\overline{\ \ \ Ricardo\ Fabbri\ \ \ }$
%\ \\\vspace{4cm}
%\center $\overline{\ \ \ Luciano\ da\ Fontoura\ Costa\ \ \ }$
\end{document}
